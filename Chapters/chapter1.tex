\chapter{能带理论}
\begin{define}
    电子的能带指\textbf{晶体}电子的能级,能带理论的主要任务就是去确定固体电子的能级。
\end{define}
作用:
\begin{itemize}
    \item 阐明固体的基本性质,比如振动谱,磁有序,电导率,热导率,光学介电函数。
    \item 预测新材料,解释新现象。
\end{itemize}
能带图:横坐标为波矢(量子数),纵坐标为能量。

能带和能带之间的间隙如何产生?

本课程所作的简化:
\begin{itemize}
    \item 绝热近似:多粒子变为多电子,认为原子核处于静止状态。
    \item 单电子近似,假设所有电子的哈密顿量为:
    \begin{equation}
        H_e=\sum_{i=1}\left[-\frac{1}{2}v_e(r_i)+\sum_{R_n}\frac{1}{4\pi\varepsilon_0}\frac{e^2}{|r_i-R_n|}\right]
    \end{equation}
    多电子变为单电子问题,但是当相互作用势不能忽略时,不能用单电子近似,比如超导现象。
    \item  周期场近似,假设所有晶体的分布一致。忽略晶格振动和晶体缺陷
\end{itemize}

\section{一维周期势场的电子运动}
\subsection{电子运动的方程}
周期性势场:
\begin{equation}
    V(x)=V(x+ni).
\end{equation}
运动方程:薛定谔方程(一维)

此处使用微扰法,电子的能量和波函数分别近似到二级和一级。

假设微扰项为:
\begin{equation}
    \Delta V=V(x)-\bar{V}.
\end{equation}
\begin{itemize}
    \item 一级微扰方程
    \item 零级波函数
    \item 边界条件与波矢的取值
    \begin{equation}
        k=\frac{2\pi l}{aN'}, l=0,\pm 1,\pm 2\cdots .
    \end{equation}
    当$N'$原胞数足够大时,可视为准连续\footnote{布里渊区的宽度与$a$晶体结构有关。}。
\end{itemize}

\subsection[Bloch Theorem]{布洛赫定理}
    当势场具有周期性边界条件(晶格势场),波函数满足满足如下性质
    \begin{equation}
        \varphi(r+\vec{R}_n)=e^{i\vec{k}\cdot \vec{R}_n}\varphi(r).
    \end{equation}
    其中$\vec{k}$为一矢量,其表明平移晶格矢量$\vec{R}_n$时,波函数只增加相位因子$e^{i\vec{k}\cdot \vec{R}_n}$。

\subsection{微扰计算(近自由电子近似)}
\begin{itemize}
    \item 微扰计算公式
    \subitem 能量微扰项
    \subitem 波函数微扰项 
    \item 微扰矩阵元,得出选择定则:K态只能散射到相差一个倒格矢$k_n=n\frac{2\pi}{a}$的位置。
\end{itemize}
如果$K=n\frac{\pi}{a}$出现简并情况,非简并微扰法不再适用,应当使用简并微扰法。

当波矢$k=n\frac{\pi}{a}$,符合选择定则的两个态进行线性组合
\begin{equation}
    \Psi^{(0)}=A\Psi^{(0)}_k+B\Psi^{(0)}_{k'}.
\end{equation}

除此之外,波矢接近Bragg反射条件的两个态的能量非常接近。借助简并微扰的思想,我们将能量很接近的两个线性组合
使其同时满足选择定则和Bragg反射条件。所得能量为:
\begin{align}
    E=T_n(1+\Delta^2)\pm\sqrt{4T^2_n\Delta^2+|V_n|^2},
    T_n= \frac{\hbar^2}{2m}\left(n\frac{\pi}{a}\right)^2.\\
\end{align}
结果讨论:
\begin{itemize}
    \item 在$k=\pm n\frac{\pi}{a}$此时$\Delta$很小
\end{itemize}
\subsection{能带与禁带}
能量本征值$E_k$是波矢$k$的函数,在零级近似,既有电子模型下的能谱为抛物线关系:
\begin{equation*}
    E_k^{(0)}=\frac{\hbar^2k^2}{2m}.
\end{equation*}
计入周期场的微扰作用后,能量在k空间倒格矢的中点,即$k=n\frac{\pi}{a}$处断开,
电子能量不能取值,禁带宽度为$2|V_n|$。

\section{三维周期场中的电子运动}
\subsection{模型和微扰计算}
\subsubsection{运动方程}
电子受到粒子周期性势场的作用,势场的起伏较小, 零级近似,用势场的平均值代替离子产生的势场
\begin{align}
    \text{势场的平均值},&\bar{V}=\frac{1}{\Omega}\int_{\Omega}V(\vec{r})\dif \vec{r},\\
    \text{周期性势场起伏量},&V(\vec{r})-\bar{V}=\Delta V.
\end{align}

\subsection{布里渊区和能带}

\subsection{紧束缚近似}
两原子接近时,核与电子之间的库仑力使电子能级分立,然后形成能带。晶体内不同量子数的原子处于不同的能带
能带的宽度与近邻原子的重叠电子云的交互作用成正比。