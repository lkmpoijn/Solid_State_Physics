\chapter{Problems}
\section{能带理论}
内容来自课后习题和习题课。
\subsection{量子力学基础}
    该题来源于黄昆书P582.
    \begin{example}
        根据$k=\pm \frac{\pi}{a}$状态\textbf{简并微扰}结果,求出与$E_+$、$E_-$对应的本征态波函数$\psi_+$和$\psi_-$,说明他们都代表驻波
        并比较两个电子云分布(即$|\psi|^2$),说明能隙的来源,假设($V_n=V_n^*$)。

        根据题意可知,该体系存在二重简并,波函数$\psi$可视为两简并态叠加
        \begin{equation*}
            \psi=A\psi_k+B\psi_{k'}.
        \end{equation*}
        带入薛定谔方程,并分别左乘两者共轭波函数可得:
        \begin{equation*}
            \left\{\begin{aligned}
                A(E_k^0-E)+BV_n^*=0,\\
                AV_n+B(E_{k'}^0-E)=0.
            \end{aligned}\right.
        \end{equation*}
        $E_+=E_k+V_n,E_-=E_{k'}-V_n$,波函数为(此处将$E$分别取$E_+$、$E_-$代入)
        \begin{equation*}
            \left\{
                \begin{aligned}
                    \psi_+=\frac{A}{\sqrt{L}}[e^{i\frac{n\pi}{a}x}-e^{-i\frac{n\pi}{a}x}]=\frac{2A}{L}\sin{\frac{n\pi}{a}x},\\
                    \psi_-=\frac{A}{\sqrt{L}}[e^{i\frac{n\pi}{a}x}+e^{-i\frac{n\pi}{a}x}]=\frac{2A}{L}\cos{\frac{n\pi}{a}x}.\\
                \end{aligned}\right.
        \end{equation*}
        在该状态下,波函数与时间无关,因此波函数为驻波。在无微扰的情况下,能量为与波矢有关的函数$E=\frac{\hbar^2k^2}{2m}$,
        微扰后,在$k=\pm \frac{\pi}{a}$的位置,能级突变$2|V_n|$,导致能隙产生。
        
        注意简并微扰,利用久期方程(secular equation)计算能量,以及 
    \end{example}
    黄昆书习题4.7
    \begin{example}
        有一一维
    \end{example}
    作业有5.1,5.2 5.4,5.6共四道题。
    2.1,2.2,2.3,2.5,2.6共5道题,14周周一交。